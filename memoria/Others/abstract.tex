\chapter*{Resumen}
	\addcontentsline{toc}{chapter}{Resumen} %evitamos que se numere
	
	Es una buena práctica intentar tener unificadas en una aplicación múltiples funcionalidades.
	Sin embargo hay ocasiones en las que esto no es posible, o poco conveniente.
	En estos casos nos vemos obligados a usar más de una aplicación que necesitan comunicarse y relacionarse entre sí.\\
	
	En estas situaciones, a pesar de que las aplicaciones suelen proporcionar herramientas que facilitan este trabajo de integración, todavía se necesita desarrollar esta
	caja negra que sirva de puente entre ambas aplicaciones.\\
	
	Este trabajo de fin de grado se ha realizado en colaboración con la empresa \textit{\acrfull{bnb}}.
	En este trabajo de fin de grado, se han integrado dos aplicaciones \textit{cloud}. La primera de ellas es el \acrfull{erp} \wday{}, que la empresa pretende empezar a usar.
	La segunda es el \acrfull{crm} \hs{} que se quiere continuar usando, él cual permite llevar un seguimiento de las oportunidades de negocio de la empresa.
	El trabajo consiste en el desarrollo de una Interfaz que actúe de puente entre ambas aplicaciones y conseguir la integridad de los datos entre las dos. \\
	
	
	Para ello se ha desarrollado la Interfaz de integración entre \hs{} y \wday{}. Esta Interfaz escucha los mensajes enviados desde \hs{}, los trata y realiza las operaciones de integración correspondientes en \wday{}.\\

	Adicionalmente en el trabajo se ha realizado un estudio de la plantilla de empresa.
	Con las conclusiones de este estudio y usando técnicas de \textit{Machine Learning} se ha construido
	un modelo basado en la información de los empleados. Este modelo es capaz de predecir futuros casos de empleados que abandonan el trabajo.
	

	\
	
	\textbf{Palabras clave}
    
    Workday, Hubspot, Integración, \acrfull{erp}, \acrfull{crm}, Cloud, Análisis de datos, Machine learning.


\chapter*{Abstract}
	\addcontentsline{toc}{chapter}{Abstract} %evitamos que se numere
	
	It's a good practice to have several functionalities in one application, instead of having them distributed in multiple ones. But there are some cases in which
	it’s not possible to accomplish that, or maybe it's simply not recommended. In these situations we are
	forced to use multiple applications that should be able to interact together.\\

	Applications usually provide services to make these integrations easier. However a lot of work is still necessary, in order to build this black box that will function as bridge between the applications.\\
	
	This project has been carried out in collaboration with the company \acrfull{bnb}. In this final degree project, I have integrated two cloud applications. One of
	them is \wday{} a relatively recent cloud based \acrshort{erp}. And the other is \hs{} a cloud based \acrshort{crm}.
	Hubspot allows users to keep track of the different business opportunities a company has.\\
	
	I have developed the Interface \textit{HubSpot-Workday} that listens to the automatic messages sent by Hubspot,
	processes them and then delivers the corresponding messages to \wday{}. When a deal or business
	opportunity is created or modified in \hs{}, our interface receives and processes the message, and then sends \wday{} requests to execute the corresponding operations.\\

	As a complementary part of the project, I have studied the data of a company’s personnel. 
	With the conclusion of this study and different techniques of \textit{Machine Learning}, I have
	built a model able to predict future cases of job attrition based on the employee's information.\\

	
	
	
	\
	
	\textbf{Keywords}
    
    Workday, Hubspot, Integration, \acrfull{erp}, \acrfull{crm}, Cloud, Data analysis, Machine learning.
	
	
