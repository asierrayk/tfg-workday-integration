\chapter{Trabajo relacionado}


\section[Enterprise Resource Planning]{Enterprise Resource Planning}

Un \acrfull{erp} o Sistema de planificación de recursos empresariales es una aplicación usada para recoger, almacenar,
interpretar y tratar grandes cantidades de información sobre actividades empresariales.
Podríamos decir que un \acrshort{erp} reúne en una aplicación las funcionalidades necesarias para desarrollar las distintas actividades empresariales de una organización o empresa.\\

Además este sistema ayuda a evaluar y controlar de manera más fácil un negocio. También permite la automatización de tareas repetitivas y una mejor comunicación entre las distintas áreas que componen la empresa.

Los \acrshort{erp} brindan información que permite a las empresas tomar decisiones que ayuden a cumplir los objetivos.\\

Sobre los años 60 y principios de los 70, antes de que los \acrshort{erp} surgieran, los sistemas utilizados por las empresas eran los llamados \acrshort{mrp} y \acrshort{mrp2}.
Los \acrshort{mrp} \footnote{Material Requirements Planning o Planificación de necesidades de material} son sistemas de planificación y administración de la producción de una empresa. Estos tipos de sistemas se centraban en aspectos relacionados con el inventario y no llegaban a abarcar muchas de las áreas importantes en una empresa.\\



El crecimiento de la información y las tecnologías de comunicaciones llevó a la creación de sistemas más robustos para 
las organizaciones. A finales de los años 80 y principios de los 90 comenzó a aparecer en el mercado un nuevo sistema 
software conocido como \acrshort{erp}, destinado principalmente a grandes y complejas organizaciones de negocios. Estos sistemas 
se caracterizaban por ser propietarios, caros, complejos y con gran potencial.
Los ERPs son sistemas que requieren de consultoría para implementar soluciones a medida, que se basen en las necesidades de la empresa u organización. \cite{hossain_rashid_patrick_2003}

Los \acrshort{erp} no surgieron de la noche a la mañana, este proceso llevó tiempo y todavía siguen evolucionando. Si por algo se han caracterizado los \acrshort{erp} son por su transformación y continuo cambio.


\subsection{SAP ERP}


\textit{SAP SE} (\textit{Systems, Applications and Products in Data Processing}) es una compañía alemana fundada en 1972. Conocida principalmente por el desarrollo del \acrshort{erp} \textit{SAP}.

\textit{SAP} fue construido partiendo de un \textit{software} previo, llamado \textit{SAP R/3} y fue lanzado el 6 de julio de 1992.
En torno al año 2000 \textit{SAP} se convirtió en la tercera compañía más grande dedicada al sector de los \textit{ERPs}.\\

Sus principales objetivos es dar soluciones a empresas pertenecientes a los sectores químico, servicios públicos, la venta al por menor, farmaceúticas\ldots\\

Los puntos fuertes de \textit{SAP} com \acrshort{erp} es proporcionar un software a las fábricas y a la parte administrativa de las empresas.

El \acrshort{erp} \textit{SAP} se trata de un software \textit{on-premise} \footnote{Software On-premise es instalado y ejecutado en el edificio (\textit{on the premises}) de la persona u organización que hace uso del software.}.
Sin embargo uno de sus enfoques está siendo dar importancia a las soluciones \textit{cloud}, y prueba de ello es el proyecto \textit{SAP HANA}.\\



Un inconveniente es que su implantación supone un riesgo, por la inversión económica y la transición a la nueva aplicación. La amortización de \textit{SAP} puede llevar mucho tiempo.

\subsection{PeopleSoft}
\textit{PeopleSoft} fue fundado por David Duffield en 1987 y adquirido por \textit{Oracle} en 2005. Es un \acrshort{erp} moderno especializado en los módulos de recursos humanos y finanzas.\\

\textit{PeopleSoft} es un \acrshort{erp} \gls{on-premise}.
Originalmente \textit{PeopleSoft} ofrecía soluciones para recursos humanos y finanzas.\\

\textit{Peoplesoft} permite tener las funcionalidades separadas en módulos independientes, pero también facilita que estos módulos trabajen conjuntamente.
A diferencia de \textit{SAP}, \textit{PeopleSoft} puede ser usado para la implementación de un único módulo, como por ejemplo administración de estudiantes o gestión de capital humano.\\

En sus últimas versiones cuenta con FLUID, que aporta la posibilidad de tener una interfaz de usuario \textit{responsive} dedicada especialmente a dispositivos móviles y \textit{tablets}.

\cite{anderson_2001}

\subsection{Workday}
\wday{} fue fundado en 2005 por David Duffield, fundador y actual \acrshort{ceo} del \acrshort{erp} \textit{PeopleSoft}.\\



Se trata de un \acrshort{erp} totalmente \textit{Cloud}. Permite a las empresas y organizaciones despreocuparse de los equipos informáticos que ejecutan el software. Se trata claramente de un aspecto diferenciador frente a \acrshort{erp} \textit{on-premise} como \textit{Peoplesoft}. 
\wday{} está destinado al mismo tipo de clientes que \textit{SAP}, y por tanto se trata de una de las alternativas.\\


\wday{} abarca los campos de finanzas y recursos humanos. Cuenta con una interfaz simple y moderna, con facilidades para su uso en dispositivos móviles.

Con \wday{} siempre se hace uso de la última versión y va siendo actualizado constantemente.


\section{Customer Relationship Management}

Un \acrfull{crm} es una aplicación o sistema usado para satisfacer las necesidades de los clientes durante cualquier interacción con los mismos.

Los \acrshort{crm} ayudan a las empresas a aumentar su rentabilidad e ingresos, logrando atraer y retener clientes de manera óptima. 
Permiten un seguimiento de las relaciones de la empresa con los clientes, así como las reuniones, acuerdos\ldots


\subsection{Salesforce}
La empresa \textit{Salesforce} fue fundada en marzo de 1999 por un antiguo ejecutivo de Oracle Marc Benioff. La empresa es conocida por su famoso \acrshort{crm}
\textit{Salesforce}. \textit{Salesforce} es un \acrshort{crm} totalmente \textit{cloud}. Por tanto se puede acceder a \textit{Salesforce} desde un navegador. Gracias a \textit{Salesforce} se consiguen automatizar numerosos procesos que ayudan a administrar las relaciones con contactos y empresas.

La configuración inicial del \acrshort{crm} \textit{Salesforce} puede ser complicado y requerir gran cantidad de tiempo.


\subsection{HubSpot}

\hs{} es un \acrshort{crm} fundado por Brian Halligan y Dharmesh Shah en 2006. Su objetivo está orientado a proporcionar herramientas para aumentar la rentabilidad de las oportunidades con los clientes. Proporciona mecanismos para la integración con otras herramientas como el correo electrónico o las redes sociales en el ámbito empresarial. \hs{} también actua como gestor de contenido y como plataforma para poder analizar los datos.

Inicialmente está pensado para pequeñas empresas, pero poco a poco se ha ido adaptando a empresas más grandes.
