\chapter{Migración de datos a Workday}

\section{Workday}


\subsection{Planteamiento de la migración}

Antes de poder integrar Workday con otros sistemas, debemos de importar datos en el portal de Workday. En esta primera tarea tenemos que pasar datos de un archivo csv a Workday. Para ello barajamos las distintas posibilidades que tenemos:
Hacer uso del EIB el cual está totalmente integrado en Workday y no hace uso de software de terceros, y principalmente se usa para integraciones pequeñas. Este método tiene la desventaja de que tienes que tener los datos en un excel con un formato específico.
Otra opción que es la que finalmente tomamos es la de integrar estos datos mediante web services. Con este método se envia a un endpoint del tenant de Workday una petición SOAP que incluye la operación que se desea realizar, la información que se pretende integrar, asi como las credenciales necesarias para hacer esa operación en dicho tenant.
 

%********************************** %Second Section  *************************************
\subsection{Pasos de la migración}

Con varios scripts en python conseguimos migrar los datos.
El primero de ellos insert\_mysql.py introduce los datos del csv en una base local mySQL.
El segundo soap.py tiene como premisa que los datos se encuentran en la base de datos local, y a partir de sus datos genera un archivo xml.
Mediante transformaciones xslt, transforma el xml en una petición SOAP con la información necesaria.
Finalmente esta petición SOAP se envía a Workday y los datos quedan integrados.

%\cite{einstein}

%\nomenclature[z-csv]{$CSV$}{Comma Separated Values}
%\nomenclature[z-soap]{$SOAP$}{Simple Object Access Protocol}





%%%%%%ANOTACIONES
%EIB uses no third-party software or hardware. 
%the purpose of the EIB, is to allow customers to build their own integrations according to their unique business scenarios.
%The Enterprise Interface Builder can be used for both exports and imports.

%INBOUND: Attachment, rest-based URL, a file transfered from an external SFTP, FTP/SSL, or FTP endpoint.