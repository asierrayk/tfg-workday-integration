
\chapter{Sistemas de Integración Workday}

\ifpdf
    \graphicspath{{Chapter2/Figs/Raster/}{Chapter2/Figs/PDF/}{Chapter2/Figs/}}
\else
    \graphicspath{{Chapter2/Figs/Vector/}{Chapter2/Figs/}}
\fi

A continuación se describirán los mecanismos utilizados para la integración de datos en Workday.

\section{Arquitectura de integración}
Los servicios de negocio se sustentan en un Object Management System (OMS). 
Todos los datos están almacenados de manera persistente y siendo accesibles únicamente a través del OMS.
Este concepto es clave para entender el funcionamiento de Workday.

Toda comunicación con el OMS se realiza por HTTP con ficheros XML. Los usuarios acceden a Workday desde sus navegadores y este es el que se comunica con el OMS con ficheros XML mediante web service requests para mostrar un HTML por la interfaz de usuario.

Los desarrolladores también pueden conectar con el OMS a través de la interfaz de usuario de servidor para hacer uso de las browser-based tools.

%TODO Grafico con la arquitectura


\section{¿Qué es EIB?}
EIB o Enterprise Interface Builder permite que los usuarios ejecuten integraciones seguras y simples en Workday.
EIB no hace uso de software de terceros y tiene como propósito es permitir a los clientes construir sus propias integraciones con el EIB.


\section{Inbound EIB}

Como complemento del EIB, Workday proporciona plantillas de hojas de cálculo para ayudar con la carga de datos en Workday. Estas hojas de cálculo se pueden generar directamente desde Workday.

Las hojas de cálculo son utiles cuando no se cargan grandes cantidades de datos. Ya que en estos casos rellenar las hojas de cálculo puede ser tedioso.

Por otro lado, información almacenada en un archivo XML (que no haya sido cargada en una plantilla de hoja de cálculo de Workday) puede ser cargada mediante el EIB.
Además existe la opción de subir un archivo XSLT para transformar la estructura del XML y encaje con el formato requerido por Workday.

Workday proporciona Web Service Description Language (WSDL) que es un  fichero XML usado para describir las operaciones que soporta el servicio.
Por tanto el fichero XML que finalmente llega Workday debe cumplir estos requisitos.

%TODO esquema

\section{Outbound EIB}

Existen dos alternativas como origen de la operación.
\begin{itemize} 
\item Reports-as-a-Service

Workday pone a disposición informes personalizados avanzados como servicios web para ser llamados en integraciones.
\item Workday Public Web Service (WWS)

\end{itemize}
Opcionalmente se puede transformar la salida del XML.
Finalmente la entrega se puede especificar como una de estas opciones.
\begin{itemize}
\item Archivo adjunto Workday
\item Protocolos de transporte externo: SFTP, FTP/SSL, FTP,  HTTP/SSL, Email, AS2, WebDAV.
\item Usando un método de entrega existente
\end{itemize}
Outbound EIB también cuenta con operaciones que pueden ser invocadas mediante servicios web de workday (WWS).

