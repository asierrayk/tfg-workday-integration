\chapter{Introducción}

\section{Antecedentes o Trabajos relacionados}


\section[ERP's]{Enterprise Planning}

Un \gls{erp} o Sistema de planificación de recursos empresariales es una aplicacion usada para recoger, almacenar, interpretar y tratar grandes cantidades de información sobre actividades empresariales.
Podríamos decir que un ERP reune en una aplicación las funcionalidades necesarias para desarrollar las distintas actividades empresariales de una organización o empresa.

\subsection{SAP}
SAP es un ERP desarrollado por una compañia alemana. SAP abarca las funcionalidades de negocio claves de una organización.

TODO (Resumen SAP)

\subsection{PeopleSoft}
PeopleSoft fue fundado por David Duffield en 1987 y adquirido por Oracle en 2005.
Originalmente PeopleSoft ofrecía soluciones para recursos humanos y finanzas.
PeopleSoft es un ERP que proporciona software de gestión de recursos humanos (\acrshort{hrms}), gestión de soluciones financieras (\acrshort{fms}),
Administración de la cadena de suministro (\acrshort{scm}),  \acrfull{crm}, Enterprise Performance Management (\acrshort{epm}).
También aporta soluciones para de administración de centro de estudios, organizaciones o gobiernos.

Peoplesoft permite tener las funcionalidades separadas en modulos independientes, pero también facilita que estos módulos trabajen conjuntamente.
A diferencia de SAP, PeopleSoft puede ser usado para la implementación de único módulo, como por ejemplo Administración de estudiantes o gestión de capital humano (hcm).
Es facilmente adaptable, a diferencia de SAP que es más concreto.

En sus últimas versiones cuenta con FLUID. Que aporta la posibilidad de tener una interfaz de usuario responsive dedicada especialmente a dispositivos móviles y tablets.

TODO (Continuar y corregir PeopleSoft)

\subsection{Workday}
Workday fue fundado en 2005 por David Duffield, fundador y actual \acrshort{ceo} del ERP PeopleSoft. Workday está destinado al mismo tipo de clientes que SAP, y por tanto se trata de una de las alternativas.

TODO (Resumen workday)



\section{Objetivos}


\section{Plan de trabajo}