\chapter{Introducción}


\section[ERP's]{Enterprise Resource Planning}

Un \gls{erp} o Sistema de planificación de recursos empresariales es una aplicación usada para recoger, almacenar,
interpretar y tratar grandes cantidades de información sobre actividades empresariales.
Podríamos decir que un ERP reúne en una aplicación las funcionalidades necesarias para desarrollar las distintas actividades empresariales de una organización o empresa.

El crecimiento de la información y las tecnologías de comunicaciones a llevado a la creación de sistemas más robustos para 
distintas organizaciones. A finales de los años 80 y principios de los 90 comenzó a aparecer en el mercado un nuevo sistema 
software conocido como \acrshort{erp}, destinado principalmente a grandes y complejas organizaciones de negocios. Estos sistemas 
se caracterizaban por ser propietarios, caros, complejos y poderosos.
Los ERPs son sistemas que requieren de consultoría para implementar soluciones medida, que se basen en las necesidades de la empresa u organización. \cite{erp1}


\subsection{SAP}


SAP o  Systems, Applications and Products in Data Processing es un ERP desarrollado por una compañía alemana, fue fundada en 1972
En torno al año 2000 SAP se convirtió en la tercera compañia más grande dedicada al sector de los ERPs.


es un ERP desarrollado por una compañia alemana. SAP abarca las funcionalidades de negocio claves de una organización.

\subsection{PeopleSoft}
PeopleSoft fue fundado por David Duffield en 1987 y adquirido por Oracle en 2005. Es uno de los ERP más modernos especializado en los módulos de recursos humanos y servicios financieros.
Originalmente PeopleSoft ofrecía soluciones para recursos humanos y finanzas.

Peoplesoft permite tener las funcionalidades separadas en módulos independientes, pero también facilita que estos módulos trabajen conjuntamente.
A diferencia de SAP, PeopleSoft puede ser usado para la implementación de un único módulo, como por ejemplo Administración de estudiantes o gestión de capital humano (hcm).
Es fácilmente adaptable, a diferencia de SAP que es más concreto.

En sus últimas versiones cuenta con FLUID. Que aporta la posibilidad de tener una interfaz de usuario responsive dedicada especialmente a dispositivos móviles y tablets.



\subsection{Workday}
Workday fue fundado en 2005 por David Duffield, fundador y actual \acrshort{ceo} del ERP PeopleSoft.

Se trata de un ERP totalmente Cloud
Workday está destinado al mismo tipo de clientes que SAP, y por tanto se trata de una de las alternativas.


\section{\acrfull{crm}}

Un \acrshort{crm} es una aplicación o sistema usado para satisfacer las necesidades de los clientes durante cualquier interacción con los mismos.

Los \acrshort{crm} ayudan a las empresas a aumentar su rentabilidad e ingresos, logrando atraer y retener clientes de manera óptima. 
Permiten un seguimiento de las relaciones de la empresa con lso clientes, as;i como reuniones, acuerdos \ldots


\subsection{Salesforce}
Se trata de un \acrshort{crm} que opera en la nube. Se trata de una de las empresas más valoradas que operan en la nube.





\section{Objetivos}


\section{Plan de trabajo}

Por parte de la empresa se dio una especificación general de las funcionalidades con las que debía contar la Interfaz e inicialmente está  especificación  estaba abierta a posibles cambios. Paralelamente al desarrollo de la Interfaz se mantuvieron reuniones periodicas con un grupo reducido de personas.

En estas reuniones se hacía una evaluación del progreso del desarrollo de la Interfaz. Se intentaba enfocar los problemas para su posible resolución.

Según iba pasando el tiempo y se acercaba la fecha de salida producción se fueron concretando muco más los objetivos, hasta el punto de llagar a la especificacion final.


para el desarrollo de la interfaz se ha usado una metodología ágil. La interfaz se ha ido desarrollando de manera iterativa e incremental. 
En cada reunión se comprobaban los avances, y comprobaban las nuevas funcionalidades.
 También se proponíanlos ojetivos a cumplir para la próximo encuentro.