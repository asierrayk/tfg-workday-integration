\chapter{Introducción}


\section{Motivaciones}
Este trabajo de fin de grado se ha realizado en colaboración con la empresa \acrfull{bnb}.
\acrshort{bnb} es una empresa de consultoría que proporciona a los clientes soluciones para distintas aplicaciones de negocio: \textit{SAP}, \textit{Workday}, \textit{PeopleSoft}, \textit{JD Edwards}.\\

Este trabajo de fin de grado es parte de un proyecto, concretamente es una de las integraciones de el proyecto.
El proyecto parte de la idea de mejorar el \textit{software} que se usa internamente para la parte de recursos humanos y finanzas. Se requiere facilitar las labores internas.
Comenzar a usar un \acrshort{erp} es una forma de conseguir reunir las funcionalidades de recursos humanos y finanzas en una aplicación.
Dado que \wday{} es un \acrshort{erp} moderno y opera en la nube, se trata de la opción que más encaja para la empresa y la que se decidió.
\\

Una vez fijado el proyecto surgen infinidad de tareas. Configurar \wday{} con todas las opciones y funcionalidades que desea tener la empresa, migrar los datos de aplicaciones previas como \textit{QlikView}, \textit{CM Plan}, 
integrar \wday{} con otras aplicaciones con las que tiene que coexistir cuando se salga a producción \ldots\\

Las aplicaciones que se deben integrar con \wday{} son \textit{Mantis Bug Tracker} y \hs{}.

Este trabajo de fin de grado consiste en la integración de \wday{} con \hs{}.

\hs{} es el \acrshort{crm} que usa actualmente la empresa y quiere mantenerlo tras la salida a producción.\\


El proceso de salida a producción debe ser rápido, ya que la empresa debe continuar con sus actividades empresariales. 
Es decir los procesos de migración y puesta en marcha de las integraciones tienen que hacerse en los dias previos a la slaida a producción.
Por ello , uno de los grandes retos que supone el proyecto es la salida a producción con el menor impacto posible a las actividades de la empresa.\\



Adicionalmente, por parte de la empresa se me solicitó hacer un prototipo de aplicación capaz de predecir la rotación de plantilla de la empresa.
Y así ser capaz de preveer cuando un empleado va a abandonar la empresa. \\

Con el desarrollo de este prototipo se quiere tomar las acciones necesarias para realizar una mejor predicción.



\section{Objetivos}

En esta sección vamos a establecer los objetivos a cumplir.

\begin{itemize}
	\item Desarrollar un microservicio que realice la integración entre \hs{} y \wday{}.
	\item Conseguir que al introducir datos en \hs{}, automaticamente se sincronicen con \wday{}.
	\item Que al modificar datos en \hs{}, se modifiquen en \wday{} de forma automática.
	\item Que la integración sea rápida, y el tiempo transcurrido entre la introducción o cambio de datos en \hs{} y los cambios en \wday{} sea el mínimo posible.
	\item La integración ha de ser segura. Estar provista de mecanismos para evitar posible ataques de terceras personas.
	\item La integración debe ser totalmente transparente para el usuario. 
	\item El programa no debe abortar su ejecución de manera inexperada.
	\item En la medida de lo posible evitar errores en la introducción de datos. Evitar que se cree información duplicada.
	\item El servicio que realiza la integración tiene que estar en ejcución ininterrumpida.
	\item Evitar la perdida de datos.
	\item Localmente se debe llevar la cuenta de los datos que se encuentran integrados.
	\item El servicio debe soportar ser reiniciado.
	\item El prototipo predictor debe ser capaz de decidir si es muy probable que un empleado abandone la empresa.
	\item Concluir que datos son más importantes para predecir el abandono en el trabajo.
	\item Poder tomar mediadas de acuerdo a las conclusiones a las que se llegue par elaborar mejores predicciones.
\end{itemize}


\section{Plan de trabajo}

Para la realización del proyecto de transición a \wday{} se ha seguido un sistema de metodología ágil. 
Cinco personas formabamos parte del equipo de proyecto y nos reuniamos al menos una vez por semana.
El proceso a durado unos tres meses y todos los integrantes realizabamos paralelamente tareas en otros proyectos.\\

En estas reuniones, se informaba de donde nos encontrabamos y cuales eran los siguientes pasos.
Por turnos se intervenía para explicar los avances que habíamos hecho hasta el momento y los problemas a los que nos estabamos enfrentando.
Las reuniones también servían para tomar decisiones sobre el proyecto.\\

En cuanto al desarrollo de la Interfaz de integración entre \hs{} y \wday{} se fue realizando de forma iterativa e incremental. 
Además, al tratarse de un grupo reducido, cualquier consulta o duda sobre los requisitos de la Interfaz podía ser solventada rápidamente.

Durante las últimas semanas se planificó mucho más detalladamente las tareas de cada uno, así como el momento en el que estas tareas debían ejecutarse.
Fue en estas últimas semanas cuando se necesitó la colaboración de otros compañeros de trabajo en el proyecto.\\

El desarrollo de la Interfaz de integración entre \hs{} y \wday{} se ha realizado de forma incremental.\\

En cada reunión se comunicaban los avances realizados y se proponían los ojetivos a cumplir para la próximo encuentro. Tras la reunion se comprobaban las nuevas funcionalidades implementadas.\\

En las semanas posteriores a la salida a producción, gracias a las observaciones e incidencias anunciadas por los usuarios,
se ha continuado modificando la Interfaz. 
