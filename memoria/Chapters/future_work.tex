\chapter{Trabajo futuro}


Es probable que en un futuro se mejore la \iface{} o se incluya directamente en el sistema de \wday{}.
Para ello habría que usar la herramienta de \textit{Workday Studio}.
Aún así siempre sería necesario tener algún servicio intermedio que continúe reenviando los mensajes provenientes \hs{} hacia \wday{}. \\

A continuación se citan algunas ideas sobre el trabajo futuro que se podría realizar en la \iface{}.
\begin{itemize}
	\item Cuando un usuario introduce por error información duplicada en \hs{}, esta se sincroniza en \wday{}. 
	Un posible trabajo futuro sería diseñar algun mecanismo que evite la integración de información duplicada. 
	Por ejemplo en la creación de un nuevo \textit{deal}, buscar si existe algún \textit{deal} con el nombre igual o muy parecido y avisar al usuario para que en última instancia decida como resolver el problema.
	
	\item Implementar un proceso recurrente que compruebe la integridad de los datos entre \hs{} y \wday{}. Si algún mensaje no se recibió, el proceso debe ser capaz de detectarlo y solucionarlo.
	
	\item Diseñar alguna interfaz de usuario para poder realizar ciertas acciones mientras la \iface{} está en ejecución. Por ejemplo una interfaz de usuario web para poder lanzar procesos, visualizar información sobre la base de datos local, actualizar de forma manual la base de datos, o comprobar que todos los datos están correctamente sincronizados.
	
	
\end{itemize}

A continuación vamos a ver el trabajo futuro que podemos reali'ar con el prototipo predictor.


\begin{itemize}
	
	\item Implementar un modelo basado en los datos reales de la empresa.
	
	\item Realizar un proceso de \textit{Feature engineering} más elaborado.
	
	\item Probar estimadores más complejos que puedan aportar un mejor resultado, por ejemplo  \textit{Ensemble methods}.
	
	\item Integrar el prototipo de predicción en el sistema de \wday{}.
	


\end{itemize}



