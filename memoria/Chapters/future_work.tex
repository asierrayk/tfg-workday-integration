\chapter{Trabajo futuro}

En un futuro se mejorará la \iface{} portandola a la plataforma de desarrollo propia de \wday{}, \textit{Workday Studio}. Aún así será necesario mantener un servicio intermedio que reenvíe los mensajes desde \hs{} hacia \wday{}.\\

Hasta que finalmente se haga dicha migración será necesario realizar algunas mejoras.

A continuación se citan algunas ideas sobre el trabajo futuro que se podría realizar en la \iface{}.
\begin{itemize}
	\item Implementar una funcionalidad capaz de detectar la introducción de información duplicada. Cuando un usuario introduzca por error información duplicada en \hs{} se debe notificar al usuario sobre el problema. De esta forma se evitaría la sincronización de información duplicada en \wday{}.
	
	Por ejemplo en la creación de un nuevo \textit{deal}, buscar si existe algún \textit{deal} con el nombre igual o muy parecido al que se esta creando y avisar al usuario para que en última instancia decida como resolver el problema.
	
	\item Realizar las pruebas necesarias en el proceso recurrente que comprueba la integridad de los datos entre \hs{} y \wday{}. De manera que pueda ser añadido al entorno de producción.
	
	\item Diseñar una interfaz de usuario para poder realizar ciertas acciones mientras la \iface{} está en ejecución. Esta interfaz de usuario por ejemplo podría ser accedida a través de una \textit{url} y permitiese lanzar procesos, visualizar información sobre la base de datos local, actualizar la base de datos manualmente o comprobar que todos los datos están correctamente sincronizados.
	
	
\end{itemize}


Ahora vamos a ver el trabajo futuro que se podría hacer en el prototipo predictor.\\

La idea principal sería implementar el modelo predictor en un entorno de producción. En dicho entorno debería operar en tiempo real, notificando con un aviso a las personas correspondiente cuando la probabilidad de abandono calculada para un empleado supere cierto umbral.

Para conseguirlo habría que ir completando una serie de tareas.

\begin{itemize}

\item Una de las primeras tareas sería desarrollar un modelo utilizando los datos reales de la empresa. Para ello podría ser necesario recabar información que se desconozca sobre los empleados. Por ejemplo la conciliación de la vida laboral con la familiar, número de empresas en las que ha estado el empleado\ldots\\

\item En este desarrollo del modelo predictor, se haría un estudio profundo sobre las características y un proceso de \textit{Feature engineering}.\\

\item Como trabajo futuro también se podría tener en cuenta más estimadores o técnicas para elegir estimadores, como por ejemplo \textit{Ensemble methods}.\\



\item Por último habría que integrar este modelo con el sistema de \wday{}. De manera que en tiempo real el modelo fuese entrenandose y realizando predicciones.



\end{itemize}


