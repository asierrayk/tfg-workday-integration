\chapter{Trabajo futuro}


Es probable que en un futuro se mejore la \iface{} o se incluya directamente en el sistema de \wday{}.
Para ello habría que usar la herramienta de \textit{Workday Studio}.
Aún así siempre sería necesario tener algún servicio que continúe reenviando los mensajes provenientes \hs{} hacia \wday{}. \\

A continuación se citan algunas ideas sobre el trabajo que se podría realizar en un futuro.
\begin{itemize}
	\item Cuando un usuario introduce por error información duplicada en \hs{}, ésta se sincroniza en \wday{}. 
	Un posible trabajo futuro sería diseñar algun mecanismo que evite la integración de información duplicada. 
	Por ejemplo en la creación de un nuevo \textit{deal}, buscar si existe algún \textit{deal} con el nombre igual o muy parecido y avisar al usuario para que en última instancia decida como resolver el problema.
	
	\item Implementar un proceso recurrente que compruebe la integridad de los datos entre \hs{} y \wday{}. Si algún mensaje no se recibió, debe ser capaz 
	
	\item Diseñar alguna interfaz de usuario para poder realizar ciertas acciones mientras la \iface{} está en ejecución. Por ejemplo una interzaz de usuario web para poder lanzar procesos para visualizar información sobre la base de datos local, o comprobar que todos los datos están correctamente sincronizados.
	
	\item Integrar el prototipo de predicción en el sistema de \wday{}.
	
	\item Tomar medidas que ayuden a recabar información necesaria para una mejor predicción del modelo. 
	Por ejemplo si la característica \textit{WorkLifeBalance} (conciliación del trabajo y la vida familiar) es importante para la predicción y en la empresa no se cuenta con dicha información, entonces se puede hacer una encuesta a los empleados para poder conseguirla.

\end{itemize}



