\chapter{Conclusiones}

El desarrollo de la \iface{} se ha realizado satisfactoriamente. Sirviendo como un proyecto de integración interno para \acrshort{bnb}.

La salida a producción de \wday{} se realizó el 4 de abril de 2017. Con ello los sistemas de integración, incluyendo la \iface{} se pusieron en funcionamiento.\\


%Integración

Con este trabajo de fin de grado he podido comprobar la versatilidad del lenguaje \textit{python} a la hora de construir micro-servicios que se comunican con varias aplicaciones.
También he podido entender que no es posible tener todo unificado en una misma aplicación. Por ello es necesario la integración de múltiples aplicaciones, para conseguir que los datos estén sincronizados.
He visto la variedad de metodologías utilizadas por distintas aplicaciones para facilitar las integraciones: servicios web, APIs\ldots
También he comprobado las diferentes formas de permitir el acceso a las funcionalidades de integración: OAuth2.0 y acceso por usuario y contraseña.\\


%Analisis de datos

En la parte del estudio de los datos de los trabajadores de una empresa, he podido comprobar las ventajas que presenta el lenguaje \textit{python} tanto para el análisis de datos como para realizar \textit{Machine Learning}.
A pesar de no haber usado lenguajes como \textit{R} o \textit{Scala} puedo asegurar que python es una buena opción cuando se trata con proyectos de \textit{Machine Learning}.\\


Ahora vamos a ir comprobando uno a uno los objetivos establecidos en el capítulo de introducción, para ver si se han cumplido.

\begin{itemize}
	\item Desarrollar un microservicio que realice la integración entre \hs{} y \wday{}.\\
	
		Este objetivo se ha conseguido.
	\item Conseguir que al introducir datos en \hs{}, automáticamente se sincronicen con \wday{}.\\
	
	Una vez que se crea un \textit{deal} en \hs{},si corresponde se crea automáticamente un proyecto en \wday{}.
	
	\item Que al modificar datos en \hs{}, se modifiquen en \wday{} de forma automática.\\
	
	Cuando un \textit{deal} es modificado en \hs{}, que tiene un \textit{project} asociado en \wday{}, el \textit{project} asociado también se ve modificado. Esta sincronización ocurre con la modificación de cualquiera de las propiedades de un \textit{deal}, incluso si modificamos el objeto \textit{company} asociado.
	
	\item Que la integración sea rápida, y el tiempo transcurrido entre la introducción o cambio de datos en \hs{} y los cambios en \wday{} sea el mínimo posible.\\
	Se ha conseguido que la integración sea instantánea gracias a las suscripciones a los \textit{webhook} de \hs{}. 
	Se ha evitado así la creación de un proceso recurrente que compruebe periódicamente si ha habido algún cambio en \hs{}. 
	
	\item La integración ha de ser segura. Estar provista de mecanismos para evitar posible ataques de terceras personas.\\
	
	Se ha conseguido garantizar la seguridad del sistema, añadiendo comprobaciones sobre los mensajes recibidos para asegurar su procedencia.
	
	\item La integración debe ser totalmente transparente para el usuario.\\
	
	El usuario no necesita saber como funciona la integración. Simplemente creando o modificando objetos en \hs{} consigue que la información se sincronice en \wday{}.
	Sin embargo los usuarios deben tener presente en que fases un \textit{deal} es sincronizado y en que fases se excluye de la sincronización.
	
	\item El programa no debe abortar su ejecución de manera inesperada.\\
	
	Gracias al \textit{framework} utilizado \textit{web.py}, no es posible que el programa finalice su ejecución de forma inesperada. 
	Ya que internamente \textit{web.py} captura todas las excepciones y las muestra por consola. En cualquier caso, se ha intentado en la medida de lo posible controlar todos los tipos de errores y excepciones que pudiesen ocurrir.
	
	\item En la medida de lo posible evitar errores en la introducción de datos. Evitar que se cree información duplicada.\\
	
	Este objetivo no se ha podido llevar a cabo, ya que la duplicación de la información en \hs{} es un error del usuario. Si el usuario no se da cuanta de la existencia de un \textit{deal} o \textit{company} puede crear uno duplicado y que consecuentemente se sincronice, cuando eso no es el resultado esperado.
	
	
	\item El servicio que realiza la integración tiene que estar en ejecución ininterrumpida.\\
	
	La \iface{} se encuentra en ejecución en un servidor externo, activo todo el tiempo.
	
	\item Evitar la pérdida de datos.\\
	
	Para evitar la pérdida de datos se ha creado un proceso que revisa todos los \textit{deals} e integraría aquellos \textit{deals} que no están integrados. 
	Ya sea debido a una interrupción del servicio o la pérdida de algún mensaje.
	
	Sin embargo no se ha podido comprobar el correcto funcionamiento de este proceso.
	\item Localmente se debe llevar la cuenta de los datos que se encuentran integrados.\\
	
	Se ha conseguido mediante el uso de una base de datos sencilla.
	Sin embargo no está totalmente protegida de errores  de los usuarios, pues por ejemplo si un \textit{deal} se pase a la fase de cierre por equivocación, se para la sincronización del mismo.
	De manera que es necesario un cambio manual en la tabla de \textit{deals} excluidos.
	
	\item El servicio debe soportar ser reiniciado.\\
	
	Gracias al almacenamiento persistente de los \textit{token}, es posible reiniciar el sistema sin problemas de funcionamiento. No obstante es posible que problemas de pérdidas de datos tengan lugar y sea necesario tomar medidas(Como por ejemplo ejecutar el proceso que revisa todos los \textit{deals}).
	\item El prototipo predictor debe ser capaz de decidir si es muy probable que un empleado abandone la empresa.\\
	
	Este objetivo se ha cumplido, aunque quizá se podría haber alcanzado mejores resultados haciendo uso de datos más fiables.
	\item Concluir que datos son más importantes para predecir el abandono en el trabajo.\\
	
	Hemos podido averiguar que datos son determinantes a la hora de predecir las rotaciones en la plantilla de una empresa.
	\item Poder tomar mediadas de acuerdo a las conclusiones a las que se llegue para elaborar mejores predicciones.\\
	
	Gracias a la información obtenida con el estudio y la predicción, se ha podido recomendar medidas de actuación que reduzcan estos sucesos en las empresas.
\end{itemize}




