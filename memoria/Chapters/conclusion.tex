\chapter{Conclusiones}

El desarrollo de la Interfaz se ha realizado satisfactoriamente. Sirviendo como un proyecto de integración interno para \acrshort{bnb}.

La salida a producción de Workday se realizó el 4 de abril de 2017. Con ello los sistemas de integración, incluyendo mi Interfaz se pusieron en funcionamiento.\\

Durante los primeros días he realizado el mantenimiento de la Interfaz así como solventado distintos problemas.\\

Es probable que en un fúturo se mejore la Interfaz o se incluya directamente en el sitema de Workday.
Para ello habría que usar las herramienta Workday Studio.
Aún así siempre sería necesario tener alun servicio que continúe reenviando los mensajees provenientes Hubspot a Workday. \\

%Integración

Con este trabajo de fin de grado he podido comprobar la versatilidad del lenguaje \textit{python} a la hora de construir microservicios que sirvan de comunicación entre varias aplicaciones.
También he podido entender que no es posible tener todo unificado en una misma aplicación. Por ello suele ser necesario la integración de multiples aplicaciones, para conseguir que los datos estén sincronizados.
He visto de primera mano cuales son las metodologías utilizadas por distintas aplicaciones para facilitar las integraciones, servicios web, APIs \ldots
Así como la forma de permitir el acceso a dichas funcionalidades: OAuth2.0 y acceso por usuario y contraseña.\\


%Analisis de datos

En la parte del estudio de los datos de los trabajadores de una empresa, he podido comprobar las ventajas que presenta el lenguaje python para tanto el analisis de datos como para realizar \textit{Machine Learning}.
A pesar de que no he usado lenguajes como R, puedo asegurar que python es una buena opción cuando se tiene proyectos relacionados con \textit{Machine Learning}.\\


Este estudio me ha servido como primer acercamiento a las técnicas utilizadas en \textit{Machine Learning} así como los estimadores y sus diferencias.
A pesar de no haber conseguido un porcentaje de acierto demasiado alto, he podido descubrir técnicas aplicables para mejorar cada vez más la precisión de nuestro modelo.\\




